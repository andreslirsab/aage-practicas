% This is samplepaper.tex, a sample chapter demonstrating the
% LLNCS macro package for Springer Computer Science proceedings;
% Version 2.21 of 2022/01/12
%
\documentclass[runningheads]{llncs}
%
\usepackage[T1]{fontenc}
% T1 fonts will be used to generate the final print and online PDFs,
% so please use T1 fonts in your manuscript whenever possible.
% Other font encondings may result in incorrect characters.
%
\usepackage{graphicx}
% Used for displaying a sample figure. If possible, figure files should
% be included in EPS format.
%
% If you use the hyperref package, please uncomment the following two lines
% to display URLs in blue roman font according to Springer's eBook style:
%\usepackage{color}
%\renewcommand\UrlFont{\color{blue}\rmfamily}
%\urlstyle{rm}
%
\begin{document}
%
\title{AAGE - Práctica 1 (MLlib): }
%
%\titlerunning{Abbreviated paper title}
% If the paper title is too long for the running head, you can set
% an abbreviated paper title here
%
\author{Andrés Lires Saborido\inst{1}\and
Ángel Vilariño García\inst{2}}
%
\authorrunning{A. Lires \and Á. Vilariño}
% First names are abbreviated in the running head.
% If there are more than two authors, 'et al.' is used.
%
\institute{ Universidade da Coruña, \email{andres.lires@udc.es} 
\and Universidade da Coruña, \email{angel.vilarino.garcia@udc.es}
}

%
\maketitle              % typeset the header of the contribution
%

\begin{abstract}
En este proyecto se presenta un problema de clasificación y predicción de retrasos de vuelos durante el año 2019, 
haciendo uso de un conjunto de datos extraído de Kaggle, \textit{2019 Airline Delays w/Weather and Airport Detail}~\cite{url_kaggle}. 
El objetivo es desarrollar un modelo de clasificación binaria capaz de estimar si un vuelo se retrasará a partir de una serie de 
variables operativas y meteorológicas conocidas antes de la salida. Dado el gran volumen y la heterogeneidad de los datos, se emplea 
Apache Spark MLlib~\cite{url_spark} para realizar todas las fases del trabajo: preprocesamiento de datos y el entrenamiento de 
diferentes modelos de aprendizaje automático a gran escala. Se comparan distintos algoritmos, así como diferentes modelos gracias 
a combinaciones de características o hiperparámetros para tratar de obtener el mejor modelo predictivo.


\keywords{Big Data \and Machine Learning \and Apache Spark \and MLlib \and Binary Classification 
\and Prediction \and Flight Delays \and Airports \and Weather \and Data Analysis \and Data Science}
\end{abstract}


\section{Introducción}
\subsection{Dataset}
El conjunto de datos elegido es \textit{2019 Airline Delays w/Weather and Airport Detail}~\cite{url_kaggle}, 
disponible en Kaggle, plataforma gratuita. Contiene más de 6 millones de vuelos realizados en 
Estados Unidos durante el año 2019, con información sobre aerolíneas, el vuelo, condiciones del 
aeropuerto, de la aeronave y meteorológicas.

El conjunto presenta 26 variables, explicadas con mayor detalle en la Tabla~\ref{tab:variables}. 
Las diferentes características, tanto categóricas como numéricas, pueden ser utilizadas de 
manera conjunta para analizar las condiciones operativas o meteorológicas que llevan al retraso 
de vuelos.


\begin{table}[h!]
\centering
\caption{Resumen de las variables del dataset}
\label{tab:variables}
\begin{tabular}{|l|c|l|}
\hline
\textbf{Variable} & \textbf{Tipo} & \textbf{Descripción} \\ \hline
MONTH & Discreta & Mes del vuelo \\ \hline
DAY\_OF\_WEEK & Discreta & Día de la semana (1=lunes) \\ \hline
DEP\_DEL15 & Binaria & Salida retrasada 15 min o más (1=sí) \\ \hline
DEP\_TIME\_BLK & Discreta & Bloque horario de salida \\ \hline
DISTANCE\_GROUP & Discreta & Grupo de distancia del vuelo \\ \hline
SEGMENT\_NUMBER & Continua & Vuelos previos del avión hoy \\ \hline
CONCURRENT\_FLIGHTS & Continua & Vuelos simultáneos en el mismo bloque \\ \hline
NUMBER\_OF\_SEATS & Continua & Número de asientos \\ \hline
CARRIER\_NAME & Continua & Aerolínea \\ \hline
AIRPORT\_FLIGHTS\_MONTH & Continua & Promedio de vuelos aeropuerto/mes \\ \hline
AIRLINE\_FLIGHTS\_MONTH & Continua & Promedio de vuelos aerolínea/mes \\ \hline
AIRLINE\_AIRPORT\_FLIGHTS\_MONTH & Continua & Promedio de vuelos aerolínea+ aeropuerto/mes \\ \hline
AVG\_MONTHLY\_PASS\_AIRPORT & Continua & Promedio pasajeros aeropuerto/mes \\ \hline
AVG\_MONTHLY\_PASS\_AIRLINE & Continua & Promedio pasajeros aerolínea/mes \\ \hline
FLT\_ATTENDANTS\_PER\_PASS & Continua & Tripulantes por pasajero \\ \hline
GROUND\_SERV\_PER\_PASS & Continua & Personal tierra por pasajero \\ \hline
PLANE\_AGE & Continua & Edad del avión \\ \hline
DEPARTING\_AIRPORT & Continua & Aeropuerto de salida \\ \hline
LATITUDE & Continua & Latitud aeropuerto \\ \hline
LONGITUDE & Continua & Longitud aeropuerto \\ \hline
PREVIOUS\_AIRPORT & Continua & Aeropuerto previo \\ \hline
PRCP & Continua & Precipitación (pulgadas) \\ \hline
SNOW & Continua & Nieve caída (pulgadas) \\ \hline
SNWD & Continua & Profundidad de nieve (pulgadas) \\ \hline
TMAX & Continua & Temp. máxima (°F) \\ \hline
AWND & Continua & Velocidad máxima viento (m/s) \\ \hline
\end{tabular}
\end{table}

El número de filas del dataset justifica el uso de Apache Spark, motor de procesamiento 
distribuido, para el entrenamiento de modelos de aprendizaje automático a gran escala.

\subsection{Definición del problema}
El objetivo del proyecto es predecir si un vuelo será retrasado o no antes de su salida, 
utilizando la información disponible en el dataset.

Se trata de un problema de clasificación binaria supervisada, donde la variable objetivo es 
DEP\_DEL15 (valor 1 si el vuelo salió con más de 15 minutos de retraso, 0 en otro caso). 
Entre las variables que nos ayudarán a lograr la predicción se encuentran características 
del vuelo (aerolínea, origen, destino, mes, día de la semana) y condiciones meteorológicas 
en el aeropuerto de origen.

Para resolver el problema se emplearán distintos modelos de Spark MLlib, haciendo uso de 
diferentes algoritmos, características o hiperparámetros. Estos modelos se evaluarán de 
acuerdo a métricas adecuadas al problema.


\section{Preprocesamiento de datos}

bla bla bla

\begin{table}
\caption{Table captions should be placed above the
tables.}\label{tab1}
\begin{tabular}{|l|l|l|}
\hline
Heading level &  Example & Font size and style\\
\hline
Title (centered) &  {\Large\bfseries Lecture Notes} & 14 point, bold\\
1st-level heading &  {\large\bfseries 1 Introduction} & 12 point, bold\\
2nd-level heading & {\bfseries 2.1 Printing Area} & 10 point, bold\\
3rd-level heading & {\bfseries Run-in Heading in Bold.} Text follows & 10 point, bold\\
4th-level heading & {\itshape Lowest Level Heading.} Text follows & 10 point, italic\\
\hline
\end{tabular}
\end{table}


\noindent Displayed equations are centered and set on a separate
line.
\begin{equation}
x + y = z
\end{equation}
Please try to avoid rasterized images for line-art diagrams and
schemas. Whenever possible, use vector graphics instead (see
Fig.~\ref{fig1}).

% \begin{figure}
% \includegraphics[width=\textwidth]{fig1.eps}
% \caption{A figure caption is always placed below the illustration.
% Please note that short captions are centered, while long ones are
% justified by the macro package automatically.} \label{fig1}
% \end{figure}

\begin{theorem}
This is a sample theorem. The run-in heading is set in bold, while
the following text appears in italics. Definitions, lemmas,
propositions, and corollaries are styled the same way.
\end{theorem}
%
% the environments 'definition', 'lemma', 'proposition', 'corollary',
% 'remark', and 'example' are defined in the LLNCS documentclass as well.
%
\begin{proof}
Proofs, examples, and remarks have the initial word in italics,
while the following text appears in normal font.
\end{proof}
For citations of references, we prefer the use of square brackets
and consecutive numbers. Citations using labels or the author/year
convention are also acceptable. The following bibliography provides
a sample reference list with entries for journal
articles~\cite{ref_article1}, an LNCS chapter~\cite{ref_lncs1}, a
book~\cite{ref_book1}, proceedings without editors~\cite{ref_proc1},
and a homepage~\cite{ref_url1}. Multiple citations are grouped
\cite{ref_article1,ref_lncs1,ref_book1},
\cite{ref_article1,ref_book1,ref_proc1,ref_url1}.

\begin{credits}
\subsubsection{\ackname} A bold run-in heading in small font size at the end of the paper is
used for general acknowledgments, for example: This study was funded
by X (grant number Y).

\subsubsection{\discintname}
It is now necessary to declare any competing interests or to specifically
state that the authors have no competing interests. Please place the
statement with a bold run-in heading in small font size beneath the
(optional) acknowledgments\footnote{If EquinOCS, our proceedings submission
system, is used, then the disclaimer can be provided directly in the system.},
for example: The authors have no competing interests to declare that are
relevant to the content of this article. Or: Author A has received research
grants from Company W. Author B has received a speaker honorarium from
Company X and owns stock in Company Y. Author C is a member of committee Z.
\end{credits}
%
% ---- Bibliography ----
%
% BibTeX users should specify bibliography style 'splncs04'.
% References will then be sorted and formatted in the correct style.
%
% \bibliographystyle{splncs04}
% \bibliography{mybibliography}
%
\begin{thebibliography}{8}
\bibitem{url_kaggle}
Página de Kaggle del dataset, \url{https://www.kaggle.com/datasets/threnjen/2019-airline-delays-and-cancellations/data}, descargado a 08/10/2025

\bibitem{url_spark}
Página sobre MLlib en la web de Spark, \url{https://spark.apache.org/mllib/}

\bibitem{ref_article1}
Author, F.: Article title. Journal \textbf{2}(5), 99--110 (2016)

\bibitem{ref_lncs1}
Author, F., Author, S.: Title of a proceedings paper. In: Editor,
F., Editor, S. (eds.) CONFERENCE 2016, LNCS, vol. 9999, pp. 1--13.
Springer, Heidelberg (2016). \doi{10.10007/1234567890}

\bibitem{ref_book1}
Author, F., Author, S., Author, T.: Book title. 2nd edn. Publisher,
Location (1999)

\bibitem{ref_proc1}
Author, A.-B.: Contribution title. In: 9th International Proceedings
on Proceedings, pp. 1--2. Publisher, Location (2010)

\bibitem{ref_url1}
LNCS Homepage, \url{http://www.springer.com/lncs}, last accessed 2023/10/25
\end{thebibliography}
\end{document}
